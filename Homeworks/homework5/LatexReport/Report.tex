\documentclass[11pt]{article}

\renewcommand*\familydefault{\sfdefault}
%%
%% to get Arial font as the sans serif font, uncomment following line:
%% \renewcommand{\sfdefault}{phv} % phv is the Arial font
\usepackage[sort,nocompress]{cite}
\usepackage[small,bf,up]{caption}
\renewcommand{\captionfont}{\footnotesize}
\usepackage[left=1in,right=1in,top=1in,bottom=1in]{geometry}
\usepackage{graphics,epsfig,graphicx,float,color}
\usepackage{latexsym,amsmath,amsthm,amssymb,epsfig,float, array}
%\usepackage{algorithm,algorithmic}
\usepackage{amsmath,amssymb,amsbsy,amsfonts,amsthm}
\usepackage{url}
\usepackage{boxedminipage}
\usepackage[sf,bf,tiny]{titlesec}
\usepackage[plainpages=false, colorlinks=true, citecolor=blue, filecolor=blue, linkcolor=blue, urlcolor=blue]{hyperref}

\usepackage{algorithmicx}
\usepackage{dsfont}
\usepackage{listings}
\usepackage{xspace}
\usepackage{algorithm}
\usepackage{algpseudocode}
\usepackage{sidecap}
\usepackage{caption}
\usepackage[numbered,framed]{matlab-prettifier}
\usepackage{pythonhighlight}
\usepackage{dsfont}
\usepackage{multirow}
\usepackage{tikz}

\usepackage{bm}
\newcommand{\uvec}[1]{\boldsymbol{\hat{\textbf{#1}}}}

\usepackage{subcaption}

\usepackage{graphics,epsfig,graphicx,float,color}
\usepackage[labelformat=parens,labelsep=quad,skip=3pt]{caption}
\usepackage{graphicx}

\lstset{
basicstyle=\small\ttfamily,
numbers=left,
numbersep=5pt,
xleftmargin=20pt,
frame=tb,
framexleftmargin=20pt
}

\renewcommand*\thelstnumber{\arabic{lstnumber}:}

\DeclareCaptionFormat{mylst}{\hrule#1#2#3}
\captionsetup[lstlisting]{format=mylst,labelfont=bf,singlelinecheck=off,labelsep=space}

\usepackage{matlab-prettifier}
\newcommand{\todo}[1]{\textcolor{red}{#1}}
% see documentation for titlesec package
% \titleformat{\section}{\large \sffamily \bfseries}
\titlelabel{\thetitle.\,\,\,}

\renewcommand{\baselinestretch}{0.994}
\newcommand{\bs}{\boldsymbol}
\newcommand{\alert}[1]{\textcolor{red}{#1}}

\setlength{\emergencystretch}{20pt}

\begin{document}

%\vspace*{.5cm}
\begin{center}
\large \textbf{%%
Fall 2022: Monte Carlo Methods}\\
\textbf{ Homework 5 }
\end{center}
\begin{center}
{\textbf{NAME:} Utkarsh Khandelwal\\}
{\textbf{Net Id:} uk2051}
\end{center}


% ---------------------------------------------------------------
\noindent We are given a 2-dimensional set of vectors $\overrightarrow{\sigma_{i}} \in R^2$ indexed by 1 dimensional periodic lattice $Z_L$
and with $||\overrightarrow{\sigma_i}||_2 = 1$. The nearest neighbor XY model of statistical physics assigns these vectors the density
$$\pi(\overrightarrow{\sigma}) = \frac{ \exp{\left( \beta \sum_{{i \leftrightarrow j}} \sigma_{\overrightarrow{i}} \sigma_{\overrightarrow{j}}\right)} }     {Z}$$

The above expression can also be simplified and written As
$$\pi(\overrightarrow{\theta}) = \frac{ \exp{\left( \beta \sum_{i \leftrightarrow j} \cos(\theta_i - \theta_j) \right)} }     {Z}$$
Here, $\theta$ is a scaler following the given density.

We have been asked to compute the consine of the angle of magnetization vector. Magnetization is defined as
$$M(\overrightarrow{\sigma}) = \sum_{i = 0}^{L - 1} \overrightarrow{\sigma_i}$$
Therefore cosine of angle of magnetization is
$$f(\overrightarrow{\sigma}) = \frac{(\overrightarrow{\sigma})_x }{(||\overrightarrow{\sigma})||}$$

In terms of angle this will be written as
$$M(\overrightarrow{\theta}) = \sum_{i = 0}^{L - 1} \cos{\theta_i} \uvec{\i} + \sin{\theta_i} \uvec{\j}$$

Therefore cosine of angle of magnetization in terms of theta is
$$f(\overrightarrow{\theta}) = \cos \left(
							\arctan \left( 
								\frac{ \sum_{i = 0}^{L - 1} \sin{\theta_i}} {\sum_{i = 0}^{L - 1} \cos{\theta_i}} 
								\right) \right)$$

\noindent \textbf{Exercise 64} asked to implement an Overdamped Langevian MCMC Secheme for both with and without metropolization step.
$$
\theta_h^{k + 1} = \theta_h^{k} + h S(\theta_h^{k}) \nabla^T \log \pi(\theta_h^{k}) + h \text{div} S(\theta_h^{k}) + \sqrt{2 h S(\theta_h^{k})} \zeta^{k + 1}
$$
So on choosing $S(\theta) = \textbf{I}$ (Identity Matrix), above equation can be simplified into
$$
\theta_h^{k + 1} = \theta_h^{k} + h \nabla^T \log \pi(\theta_h^{k}) + h \text{div} \textbf{I} + \sqrt{2 h} \textbf{I} \zeta^{k + 1}
$$
Since, divergence of a constanct is $0$, it can be further simplified into
$$
\theta_h^{k + 1} = \theta_h^{k} + h \nabla^T \log \pi(\theta_h^{k}) + \sqrt{2 h} \textbf{I} \zeta^{k + 1}
$$
Here, $\zeta$ are generated using Multivarite Gaussian Distribution with Covariance as $Identity Matrix$

On using this equation to generate the samples from distribution $\pi$ and plotting the histogram we are getting

\begin{figure}[H]
	\centering
	\begin{subfigure}{.40\textwidth}
		\includegraphics[width=\textwidth, scale=1]{/Users/utkarsh/NYU/Monte Carlo Methods/Homeworks/homework5/LatexReport/Histogram_Overdamped_notM_h_0d125_Samples_760000_beta_0d2_L_5.png}
		\caption{Without Metropolization}
	\end{subfigure}
	\begin{subfigure}{.40\textwidth}
		\includegraphics[width=\textwidth, scale=1]{/Users/utkarsh/NYU/Monte Carlo Methods/Homeworks/homework5/LatexReport/Histogram_Overdamped_M_h_0d125_Samples_760000_beta_0d2_L_5.png}
		\caption{With Metropolization}
	\end{subfigure}
	\caption{Histograms for magnetization samples generated for Overdamped Langevian MCMC Schemes}
	\label{fig:histograms_odlv}
\end{figure}
Looking, at the histogram we can validate the implementation as the the cosine of magnetization should either transitioing between $+1$ and $-1$ with almost equal probability. And
we can see that maximum sampled points are lying at the two ends of the histogram.

We can compare the quality of the samples with the IAT value of the cosine angle of magnetization. Since IAT are notoriously hard to accurately compute, I ran independent simulations on M parallel chains, with other parameters as constant.
And after computing IAT, took a mean of them.

From figure \ref{fig:iat_plots_q64} it we can conclude the following
\begin{itemize}
	\item As we decrease the value of $h$ by a factor of $2$ meant value of the IAT also increases by the same factor of $2$. 
	From the above figure we can see that $h$ decreases from $0.5$ to $0.25$ to $0.125$ the IATs for both wit and without metropolization incases in the same proportion.
	\item IAT of Overdamped Langevian with metropolization is more than the IAT without metropolization. This is because 
	of the  accept reject step in metropolization method.
	\item Due to smaller size of lattice, IAT converges even for samller sample count.
	\item There is not much of the difference in the IAT values with the variation in $L$. So we could infer that L is independent of the cosine of magnetization.
\end{itemize}


\begin{figure}[H]
	\centering

	\begin{subfigure}{.30\textwidth}
		\includegraphics[width=\textwidth, scale=1]{/Users/utkarsh/NYU/Monte Carlo Methods/Homeworks/homework5/LatexReport/Overdamped_Mean_IAT__h_0d5_L5.png}
		\caption{$h = 0.5$ and $L = 5$}
	\end{subfigure}
	\begin{subfigure}{.30\textwidth}
		\includegraphics[width=\textwidth, scale=1]{/Users/utkarsh/NYU/Monte Carlo Methods/Homeworks/homework5/LatexReport/Overdamped_Mean_IAT__h_0d25_L5.png}
		\caption{$h = 0.25$ and $L = 5$}
	\end{subfigure}
	\begin{subfigure}{.30\textwidth}
		\includegraphics[width=\textwidth, scale=1]{/Users/utkarsh/NYU/Monte Carlo Methods/Homeworks/homework5/LatexReport/Overdamped_Mean_IAT__h_0d125_L5.png}
		\caption{$h = 0.125$ and $L = 5$}
	\end{subfigure}


	\begin{subfigure}{.30\textwidth}
		\includegraphics[width=\textwidth, scale=1]{/Users/utkarsh/NYU/Monte Carlo Methods/Homeworks/homework5/LatexReport/Overdamped_Mean_IAT__h_0d5_L10.png}
		\caption{$h = 0.5$ and $L = 10$}
	\end{subfigure}
	\begin{subfigure}{.30\textwidth}
		\includegraphics[width=\textwidth, scale=1]{/Users/utkarsh/NYU/Monte Carlo Methods/Homeworks/homework5/LatexReport/Overdamped_Mean_IAT__h_0d25_L10.png}
		\caption{$h = 0.25$ and $L = 10$}
	\end{subfigure}
	\begin{subfigure}{.30\textwidth}
		\includegraphics[width=\textwidth, scale=1]{/Users/utkarsh/NYU/Monte Carlo Methods/Homeworks/homework5/LatexReport/Overdamped_Mean_IAT__h_0d125_L10.png}
		\caption{$h = 0.125$ and $L = 10$}
	\end{subfigure}

	
	\begin{subfigure}{.30\textwidth}
		\includegraphics[width=\textwidth, scale=1]{/Users/utkarsh/NYU/Monte Carlo Methods/Homeworks/homework5/LatexReport/Overdamped_Mean_IAT__h_0d5_L15.png}
		\caption{$h = 0.5$ and $L = 15$}
	\end{subfigure}
	\begin{subfigure}{.30\textwidth}
		\includegraphics[width=\textwidth, scale=1]{/Users/utkarsh/NYU/Monte Carlo Methods/Homeworks/homework5/LatexReport/Overdamped_Mean_IAT__h_0d25_L15.png}
		\caption{$h = 0.25$ and $L = 15$}
	\end{subfigure}
	\begin{subfigure}{.30\textwidth}
		\includegraphics[width=\textwidth, scale=1]{/Users/utkarsh/NYU/Monte Carlo Methods/Homeworks/homework5/LatexReport/Overdamped_Mean_IAT__h_0d125_L15.png}
		\caption{$h = 0.125$ and $L = 15$}
	\end{subfigure}
	\caption{Plots of IAT with samples generated for Overdamped Langevian MCMC for both with and without metropolization for different values of $h$ and $L$}
	\label{fig:iat_plots_q64}
\end{figure}



\noindent \textbf{Exercise 65} asked to implement an Hybrid Monte Carlo Scheme for both with and without metropolization step using Equation (5.26) and Algorithm 4
from the notes.

Following are the expressions chosen for solving Velocity Verlet Scheme
$$\hat{d} = L \text{   and     } \tilde{d} = L$$
$$J(\hat{x}) = \textbf{I}_{\hat{d} \text{x} \hat{d}}$$
$$K(\tilde{x}) = \frac{||\tilde{x}|^2_2}{2}$$
$$n = 10$$
Figure \ref{fig:histograms_q65} contains the histogram of the generated samples.

On using this equation to generate the samples from distribution $\pi$ and plotting the histogram we are getting
\begin{figure}[H]
	\centering
	\begin{subfigure}{.40\textwidth}
		\includegraphics[width=\textwidth, scale=1]{/Users/utkarsh/NYU/Monte Carlo Methods/Homeworks/homework5/LatexReport/Histogram_Hybrid_notM_h_0d125_Samples_800000_beta_0d2_L_10.png}
		\caption{Without Metropolization}
	\end{subfigure}
	\begin{subfigure}{.40\textwidth}
		\includegraphics[width=\textwidth, scale=1]{/Users/utkarsh/NYU/Monte Carlo Methods/Homeworks/homework5/LatexReport/Histogram_Hybrid_M_h_0d125_Samples_800000_beta_0d2_L_10.png}
		\caption{With Metropolization}
	\end{subfigure}
	\caption{Histograms for magnetization samples generated for Overdamped Langevian MCMC Schemes}
	\label{fig:histograms_q65}
\end{figure}

Now, while measuring the IAT we need to take care of multiplication by the factor of $n$, since for generating $1$ sample we generated $n$ points using Velocity Verlet and this was taken into accounting.
\ref{fig:iat_plots_q65} contains the plots of IATs for different value of h.

\begin{figure}[H]
	\centering
	\begin{subfigure}{.30\textwidth}
		\includegraphics[width=\textwidth, scale=1]{/Users/utkarsh/NYU/Monte Carlo Methods/Homeworks/homework5/LatexReport/Hybrid_Mean_IAT__h_0d5.png}
		\caption{$h = 0.5$ and $L = 10$}
	\end{subfigure}
	\begin{subfigure}{.30\textwidth}
		\includegraphics[width=\textwidth, scale=1]{/Users/utkarsh/NYU/Monte Carlo Methods/Homeworks/homework5/LatexReport/Hybrid_Mean_IAT__h_0d25.png}
		\caption{$h = 0.25$ and $L = 10$}
	\end{subfigure}
	\begin{subfigure}{.30\textwidth}
		\includegraphics[width=\textwidth, scale=1]{/Users/utkarsh/NYU/Monte Carlo Methods/Homeworks/homework5/LatexReport/Hybrid_Mean_IAT__h_0d125.png}
		\caption{$h = 0.125$ and $L = 15$}
	\end{subfigure}
	\caption{Plots of IAT with samples generated for Hybrid Monte Carlo for both with and without metropolization for different values of $h$}
	\label{fig:iat_plots_q65}
\end{figure}


From figure \ref{fig:iat_plots_q65}, we can conclude the following
\begin{itemize}
	\item We don't observe much difference between the IAT values for both with and without Metropolization.
	\item On decreasing the value of $h$ we observe that the value of $IAT$ are increasing.
	\item Due to smaller size of lattice, IAT converges even for samller sample count.
\end{itemize}

On comparing the figure \ref{fig:iat_plots_q64} and \ref{fig:iat_plots_q65} we can compare Overdamped Langevian MC Schemes as well as Hybrid MC Schemes. Below are the comparision observations:
\begin{itemize}
	\item In overdamped scheme there is huge difference with and without metropolization while in hybrid scheme we get almost same results with both the metropolization and without metropolization.
	\item In overdamped scheme decrease in the value of $h$ by some factor increased the value of IAT by same factor. Whereas in the hybrid sceme with observe the increase but not by the same factor.
\end{itemize}

\noindent \textbf{Exercise 66} asked us to implement Underdamped Langevian Monte Carlo
Following are the expresions used various factors and matrices.
$$\hat{d} = L \text{   and     } \tilde{d} = L$$
$$J(\hat{x}) = \textbf{I}_{\hat{d} \text{x} \hat{d}}$$
$$K(\tilde{x}) = \frac{||\tilde{x}|^2_2}{2}$$
$$H(\hat{x}, \tilde{x}) = - \log \pi(\hat{x}) + K(\tilde{x})$$

\noindent Implementation of the code was validated using the same mechanism of generating samples and plotting the histogram. It was similar in
shape as previous two parts, hence the code was validated.


\begin{figure}[H]
	\centering
	\begin{subfigure}{.45\textwidth}
		\includegraphics[width=\textwidth, scale=1]{/Users/utkarsh/NYU/Monte Carlo Methods/Homeworks/homework5/LatexReport/Underdamped_Mean_IAT__h_0d5.png}
		\caption{$h = 0.5$ and $L = 10$}
	\end{subfigure}
	\begin{subfigure}{.45\textwidth}
		\includegraphics[width=\textwidth, scale=1]{/Users/utkarsh/NYU/Monte Carlo Methods/Homeworks/homework5/LatexReport/Underdamped_Mean_IAT__h_0d25.png}
		\caption{$h = 0.25$ and $L = 10$}
	\end{subfigure}
	\begin{subfigure}{.45\textwidth}
		\includegraphics[width=\textwidth, scale=1]{/Users/utkarsh/NYU/Monte Carlo Methods/Homeworks/homework5/LatexReport/Underdamped_Mean_IAT__h_0d125.png}
		\caption{$h = 0.125$ and $L = 15$}
	\end{subfigure}
	\caption{Plots of IAT with samples generated for Hybrid Monte Carlo for both with and without metropolization for different values of $h$}
	\label{fig:iat_plots_q66}
\end{figure}


From figure \ref{fig:iat_plots_q65}, we can conclude the following
\begin{itemize}
	\item On increasing the value of $\gamma$ we see the increase in the value of IAT.
	\item Only reducing the value of h some factor, the value of IAT multiples by the same factor
\end{itemize}

\end{document}
