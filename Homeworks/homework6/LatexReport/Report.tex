\documentclass[11pt]{article}

\renewcommand*\familydefault{\sfdefault}
%%
%% to get Arial font as the sans serif font, uncomment following line:
%% \renewcommand{\sfdefault}{phv} % phv is the Arial font
\usepackage[sort,nocompress]{cite}
\usepackage[small,bf,up]{caption}
\renewcommand{\captionfont}{\footnotesize}
\usepackage[left=1in,right=1in,top=1in,bottom=1in]{geometry}
\usepackage{graphics,epsfig,graphicx,float,color}
\usepackage{latexsym,amsmath,amsthm,amssymb,epsfig,float, array}
%\usepackage{algorithm,algorithmic}
\usepackage{amsmath,amssymb,amsbsy,amsfonts,amsthm}
\usepackage{url}
\usepackage{boxedminipage}
\usepackage[sf,bf,tiny]{titlesec}
\usepackage[plainpages=false, colorlinks=true, citecolor=blue, filecolor=blue, linkcolor=blue, urlcolor=blue]{hyperref}

\usepackage{algorithmicx}
\usepackage{dsfont}
\usepackage{listings}
\usepackage{xspace}
\usepackage{algorithm}
\usepackage{algpseudocode}
\usepackage{sidecap}
\usepackage{caption}
\usepackage[numbered,framed]{matlab-prettifier}
\usepackage{pythonhighlight}
\usepackage{dsfont}
\usepackage{multirow}
\usepackage{tikz}

\usepackage{bm}
\newcommand{\uvec}[1]{\boldsymbol{\hat{\textbf{#1}}}}

\usepackage{subcaption}

\usepackage{graphics,epsfig,graphicx,float,color}
\usepackage[labelformat=parens,labelsep=quad,skip=3pt]{caption}
\usepackage{graphicx}

\lstset{
basicstyle=\small\ttfamily,
numbers=left,
numbersep=5pt,
xleftmargin=20pt,
frame=tb,
framexleftmargin=20pt
}

\renewcommand*\thelstnumber{\arabic{lstnumber}:}

\DeclareCaptionFormat{mylst}{\hrule#1#2#3}
\captionsetup[lstlisting]{format=mylst,labelfont=bf,singlelinecheck=off,labelsep=space}

\usepackage{matlab-prettifier}
\newcommand{\todo}[1]{\textcolor{red}{#1}}
% see documentation for titlesec package
% \titleformat{\section}{\large \sffamily \bfseries}
\titlelabel{\thetitle.\,\,\,}

\renewcommand{\baselinestretch}{0.994}
\newcommand{\bs}{\boldsymbol}
\newcommand{\alert}[1]{\textcolor{red}{#1}}

\setlength{\emergencystretch}{20pt}

\begin{document}

%\vspace*{.5cm}
\begin{center}
\large \textbf{%%
Fall 2022: Monte Carlo Methods}\\
\textbf{ Homework 6 }
\end{center}
\begin{center}
{\textbf{NAME:} Utkarsh Khandelwal\\}
{\textbf{Net Id:} uk2051}
\end{center}


% ---------------------------------------------------------------
\noindent \textbf{Exercise 71} and \textbf{Exercise 75} asked us to generate samples from Rosenbrock density.
$$\pi(x) \propto \exp \left(- \frac{100 (x_2 - x_1^2){^2}  + (1 - x_1)^2}{20} \right)$$

In \textbf{Exercise 71} we are asked to generated samples using the Stocastic Newton Method and Overdamped Langevin Scheme.
While solving the equation
$$
X_h^{k + 1} = X_h^{k} + h S(X_h^{k}) \nabla^T \log \pi(X_h^{k}) + h \text{div} S(X_h^{k}) + \sqrt{2 h S(X_h^{k})} \zeta^{k + 1}
$$

This algorithm was implemented such that when method was passed as newton, 
$$S = -(D^2 \log \pi(x))^{-1}$$

While soving for $S$ analytically, a region was found where Hessian was not invertible. So, calculation of S was modified to
$$S = -(D^2 \log \pi(x) + \eta * I)^{-1}$$
Here, $\eta$ is samll scaler in order to facilitate inversion and $I$ is the identity matrix.

Also, since matrix $S$ is the function of $x_1$ and $x_2$, care was taken while writing metropolization step.
Therefore proposal distribution was selected as
$$q(y |x ) = \exp\left(- \frac{ (y - x - hS(x)\nabla^T \log \pi (x) )^T S(x)^{-1} (y - x - hS(x)\nabla^T \log \pi (x) ) } {4h} \right)$$

For the measure of performance, quantity of measure that was chosen was $x_1$. So IAT was measured for the same. Total samples generated were 2million points.


Implementating these following are the reuslts obtained for Overdamped Stocastic Newton Scheme
\begin{figure}[H]
	\centering
	\begin{subfigure}{.22\textwidth}
		\includegraphics[width=\textwidth]{/Users/utkarsh/NYU/Monte Carlo Methods/Homeworks/homework6/LatexReport/ScatterPlotOf_StochasticNewton.png}
		\caption{Scatter Plot of Generated Samples}
		\label{fig:ovsns_scatter}
	\end{subfigure}
	\begin{subfigure}{.22\textwidth}
		\includegraphics[width=\textwidth]{/Users/utkarsh/NYU/Monte Carlo Methods/Homeworks/homework6/LatexReport/IAT_StochasticNewton.png}
		\caption{Plot of IAT vs Grenerated Samples Count}
		\label{fig:ovsns_iat}
	\end{subfigure}
	\begin{subfigure}{.22\textwidth}
		\includegraphics[width=\textwidth]{/Users/utkarsh/NYU/Monte Carlo Methods/Homeworks/homework6/LatexReport/TimeSeries_x1_StochasticNewton.png}
		\caption{Time series of $x_1$}
		\label{fig:ovsns_time_x1}
	\end{subfigure}
	\begin{subfigure}{.22\textwidth}
		\includegraphics[width=\textwidth]{/Users/utkarsh/NYU/Monte Carlo Methods/Homeworks/homework6/LatexReport/TimeSeries_x2_StochasticNewton.png}
		\caption{Time series of $x_2$}
		\label{fig:ovsns_time_x2}
	\end{subfigure}
	\caption{Plots generated for Overdamped Stocastic Newton Scheme}
	\label{fig:ovsns_plots}
\end{figure}

\begin{figure}[H]
	\centering
	\begin{subfigure}{.22\textwidth}
		\includegraphics[width=\textwidth]{/Users/utkarsh/NYU/Monte Carlo Methods/Homeworks/homework6/LatexReport/ScatterPlotOf_OverdampedLangevin.png}
		\caption{Scatter Plot of Generated Samples}
		\label{fig:ols_scatter}
	\end{subfigure}
	\begin{subfigure}{.22\textwidth}
		\includegraphics[width=\textwidth]{/Users/utkarsh/NYU/Monte Carlo Methods/Homeworks/homework6/LatexReport/IAT_OverdampedLangevin.png}
		\caption{Plot of IAT vs Grenerated Samples Count}
		\label{fig:ols_iat}
	\end{subfigure}
	\begin{subfigure}{.22\textwidth}
		\includegraphics[width=\textwidth]{/Users/utkarsh/NYU/Monte Carlo Methods/Homeworks/homework6/LatexReport/TimeSeries_x1_OverdampedLangevin.png}
		\caption{Time series of $x_1$}
		\label{fig:ols_time_x1}
	\end{subfigure}
	\begin{subfigure}{.22\textwidth}
		\includegraphics[width=\textwidth]{/Users/utkarsh/NYU/Monte Carlo Methods/Homeworks/homework6/LatexReport/TimeSeries_x2_OverdampedLangevin.png}
		\caption{Time series of $x_2$}
		\label{fig:ols_time_x2}
	\end{subfigure}
	\caption{Plots generated for Overdamped Stocastic Newton Scheme with $S = I$}
	\label{fig:ols_plots}
\end{figure}

Following are the observations that could be made from \ref{fig:ovsns_plots} and \ref{fig:ols_plots}
\begin{enumerate}
	\item As we are already aware of contours for Rosenbrock Function, we can easity infer that these scatter plots are far from true values. For \ref{fig:ovsns_scatter} 
	Markov Chain have explored smaller regioin of the $x_1$ coordiante axis.
	\item Those it's IAT looks converging but the reason is that in that region it deosn't know much about the Banana Shape of the density.
	\item While looking at the scatter plot of \ref{fig:ols_scatter}, generated points are having lot of noise in them but they do have covered a larger region in space.
	\item We can infer that if we increasing the generated samples count, Stochastic Newton would perform better becuase of less noise being generated than the Overdamped Langevin Scheme
\end{enumerate}


\textbf{Exercise 75} asked us to implemented the Ensemble Monte Carlo and compare its performace for various values of $\alpha$. I experimented with values of $\alpha  = \{1.5, 2, 2.5, 3\}$
For all the below figures, number of markov chains are $L = 10$.
\begin{figure}[H]
	\centering
	\begin{subfigure}{.22\textwidth}
		\includegraphics[width=\textwidth]{/Users/utkarsh/NYU/Monte Carlo Methods/Homeworks/homework6/LatexReport/ScatterPlotOf_Ensemble_alpha1d5.png}
		\caption{Scatter Plot of Generated Samples}
		\label{fig:en_1.5_scatter}
	\end{subfigure}
	\begin{subfigure}{.22\textwidth}
		\includegraphics[width=\textwidth]{/Users/utkarsh/NYU/Monte Carlo Methods/Homeworks/homework6/LatexReport/IAT_EnsembleScheme_alpha1d5.png}
		\caption{Plot of IAT vs Grenerated Samples Count}
		\label{fig:en_1.5_iat}
	\end{subfigure}
	\begin{subfigure}{.22\textwidth}
		\includegraphics[width=\textwidth]{/Users/utkarsh/NYU/Monte Carlo Methods/Homeworks/homework6/LatexReport/TimeSeries_x1_EnsembleScheme_alpha1d5.png}
		\caption{Time series of $x_1$}
		\label{fig:en_1.5_time_x1}
	\end{subfigure}
	\begin{subfigure}{.22\textwidth}
		\includegraphics[width=\textwidth]{/Users/utkarsh/NYU/Monte Carlo Methods/Homeworks/homework6/LatexReport/TimeSeries_x2_EnsembleScheme_alpha1d5.png}
		\caption{Time series of $x_2$}
		\label{fig:en_1.5_time_x2}
	\end{subfigure}
	\caption{Plots generated for Ensemble Scheme with $\alpha = 1.5$ and $L = 10$}
	\label{fig:en_1.5_plots}
\end{figure}


\begin{figure}[H]
	\centering
	\begin{subfigure}{.22\textwidth}
		\includegraphics[width=\textwidth]{/Users/utkarsh/NYU/Monte Carlo Methods/Homeworks/homework6/LatexReport/ScatterPlotOf_Ensemble_alpha2d0.png}
		\caption{Scatter Plot of Generated Samples}
		\label{fig:en_2_scatter}
	\end{subfigure}
	\begin{subfigure}{.22\textwidth}
		\includegraphics[width=\textwidth]{/Users/utkarsh/NYU/Monte Carlo Methods/Homeworks/homework6/LatexReport/IAT_EnsembleScheme_alpha2d0.png}
		\caption{Plot of IAT vs Grenerated Samples Count}
		\label{fig:en_2_iat}
	\end{subfigure}
	\begin{subfigure}{.22\textwidth}
		\includegraphics[width=\textwidth]{/Users/utkarsh/NYU/Monte Carlo Methods/Homeworks/homework6/LatexReport/TimeSeries_x1_EnsembleScheme_alpha2d0.png}
		\caption{Time series of $x_1$}
		\label{fig:en_2_time_x1}
	\end{subfigure}
	\begin{subfigure}{.22\textwidth}
		\includegraphics[width=\textwidth]{/Users/utkarsh/NYU/Monte Carlo Methods/Homeworks/homework6/LatexReport/TimeSeries_x2_EnsembleScheme_alpha2d0.png}
		\caption{Time series of $x_2$}
		\label{fig:en_2_time_x2}
	\end{subfigure}
	\caption{Plots generated for Ensemble Scheme with $\alpha = 2$ and $L = 10$}
	\label{fig:en_2_plots}
\end{figure}


\begin{figure}[H]
	\centering
	\begin{subfigure}{.22\textwidth}
		\includegraphics[width=\textwidth]{/Users/utkarsh/NYU/Monte Carlo Methods/Homeworks/homework6/LatexReport/ScatterPlotOf_Ensemble_alpha2d5.png}
		\caption{Scatter Plot of Generated Samples}
		\label{fig:en_2.5_scatter}
	\end{subfigure}
	\begin{subfigure}{.22\textwidth}
		\includegraphics[width=\textwidth]{/Users/utkarsh/NYU/Monte Carlo Methods/Homeworks/homework6/LatexReport/IAT_EnsembleScheme_alpha2d5.png}
		\caption{Plot of IAT vs Grenerated Samples Count}
		\label{fig:en_2.5_iat}
	\end{subfigure}
	\begin{subfigure}{.22\textwidth}
		\includegraphics[width=\textwidth]{/Users/utkarsh/NYU/Monte Carlo Methods/Homeworks/homework6/LatexReport/TimeSeries_x1_EnsembleScheme_alpha2d5.png}
		\caption{Time series of $x_1$}
		\label{fig:en_2.5_time_x1}
	\end{subfigure}
	\begin{subfigure}{.22\textwidth}
		\includegraphics[width=\textwidth]{/Users/utkarsh/NYU/Monte Carlo Methods/Homeworks/homework6/LatexReport/TimeSeries_x2_EnsembleScheme_alpha2d5.png}
		\caption{Time series of $x_2$}
		\label{fig:en_2.5_time_x2}
	\end{subfigure}
	\caption{Plots generated for Ensemble Scheme with $\alpha = 2.5$ and $L = 10$}
	\label{fig:en_2.5_plots}
\end{figure}




\begin{figure}[H]
	\centering
	\begin{subfigure}{.22\textwidth}
		\includegraphics[width=\textwidth]{/Users/utkarsh/NYU/Monte Carlo Methods/Homeworks/homework6/LatexReport/ScatterPlotOf_Ensemble_alpha3d0.png}
		\caption{Scatter Plot of Generated Samples}
		\label{fig:en_3_scatter}
	\end{subfigure}
	\begin{subfigure}{.22\textwidth}
		\includegraphics[width=\textwidth]{/Users/utkarsh/NYU/Monte Carlo Methods/Homeworks/homework6/LatexReport/IAT_EnsembleScheme_alpha3d0.png}
		\caption{Plot of IAT vs Grenerated Samples Count}
		\label{fig:en_3_iat}
	\end{subfigure}
	\begin{subfigure}{.22\textwidth}
		\includegraphics[width=\textwidth]{/Users/utkarsh/NYU/Monte Carlo Methods/Homeworks/homework6/LatexReport/TimeSeries_x1_EnsembleScheme_alpha3d0.png}
		\caption{Time series of $x_1$}
		\label{fig:en_3_time_x1}
	\end{subfigure}
	\begin{subfigure}{.22\textwidth}
		\includegraphics[width=\textwidth]{/Users/utkarsh/NYU/Monte Carlo Methods/Homeworks/homework6/LatexReport/TimeSeries_x2_EnsembleScheme_alpha3d0.png}
		\caption{Time series of $x_2$}
		\label{fig:en_3_time_x2}
	\end{subfigure}
	\caption{Plots generated for Ensemble Scheme with $\alpha = 3$ and $L = 10$}
	\label{fig:en_3_plots}
\end{figure}



\begin{figure}[H]
	\centering
	\begin{subfigure}{.22\textwidth}
		\includegraphics[width=\textwidth]{/Users/utkarsh/NYU/Monte Carlo Methods/Homeworks/homework6/LatexReport/ScatterPlotOf_Ensemble_alpha1d5_20.png}
		\caption{Scatter Plot of Generated Samples}
		\label{fig:en_1.5_scatter_20}
	\end{subfigure}
	\begin{subfigure}{.22\textwidth}
		\includegraphics[width=\textwidth]{/Users/utkarsh/NYU/Monte Carlo Methods/Homeworks/homework6/LatexReport/IAT_EnsembleScheme_alpha1d5_20.png}
		\caption{Plot of IAT vs Grenerated Samples Count}
		\label{fig:en_1.5_iat_20}
	\end{subfigure}
	\begin{subfigure}{.22\textwidth}
		\includegraphics[width=\textwidth]{/Users/utkarsh/NYU/Monte Carlo Methods/Homeworks/homework6/LatexReport/TimeSeries_x1_EnsembleScheme_alpha1d5_20.png}
		\caption{Time series of $x_1$}
		\label{fig:en_1.5_time_x1_20}
	\end{subfigure}
	\begin{subfigure}{.22\textwidth}
		\includegraphics[width=\textwidth]{/Users/utkarsh/NYU/Monte Carlo Methods/Homeworks/homework6/LatexReport/TimeSeries_x2_EnsembleScheme_alpha1d5_20.png}
		\caption{Time series of $x_2$}
		\label{fig:en_1.5_time_x2_20}
	\end{subfigure}
	\caption{Plots generated for Ensemble Scheme with $\alpha = 1.5$ and $L = 20$}
	\label{fig:en_1.5_plots}
\end{figure}


\begin{figure}[H]
	\centering
	\begin{subfigure}{.22\textwidth}
		\includegraphics[width=\textwidth]{/Users/utkarsh/NYU/Monte Carlo Methods/Homeworks/homework6/LatexReport/ScatterPlotOf_Ensemble_alpha2d0_20.png}
		\caption{Scatter Plot of Generated Samples}
		\label{fig:en_2_scatter_20}
	\end{subfigure}
	\begin{subfigure}{.22\textwidth}
		\includegraphics[width=\textwidth]{/Users/utkarsh/NYU/Monte Carlo Methods/Homeworks/homework6/LatexReport/IAT_EnsembleScheme_alpha2d0_20.png}
		\caption{Plot of IAT vs Grenerated Samples Count}
		\label{fig:en_2_iat_20}
	\end{subfigure}
	\begin{subfigure}{.22\textwidth}
		\includegraphics[width=\textwidth]{/Users/utkarsh/NYU/Monte Carlo Methods/Homeworks/homework6/LatexReport/TimeSeries_x1_EnsembleScheme_alpha2d0_20.png}
		\caption{Time series of $x_1$}
		\label{fig:en_2_time_x1_20}
	\end{subfigure}
	\begin{subfigure}{.22\textwidth}
		\includegraphics[width=\textwidth]{/Users/utkarsh/NYU/Monte Carlo Methods/Homeworks/homework6/LatexReport/TimeSeries_x2_EnsembleScheme_alpha2d0_20.png}
		\caption{Time series of $x_2$}
		\label{fig:en_2_time_x2_20}
	\end{subfigure}
	\caption{Plots generated for Ensemble Scheme with $\alpha = 2$ and $L = 20$}
	\label{fig:en_2_plots_20}
\end{figure}


Following are the observations that could be made from \ref{fig:en_1.5_plots}, \ref{fig:en_2_plots}, \ref{fig:en_2.5_plots}, \ref{fig:en_3_plots}
\begin{enumerate}
	\item Figure \ref{fig:en_2_scatter} shows the best generated plots for samples from the Rosenbrock density which is generate using $\alpha = 2$.
	\item Figure \ref{fig:en_1.5_scatter}, for $\alpha = 1.5$ genrated points, have losts of noise in it. While figure \ref{fig:en_2.5_scatter}, for $\alpha = 2$ and figure \ref{fig:en_3_scatter}, for $\alpha = 3$ have generated
	2 million samples in a very confined region. So it could be inferred that there is surely an optimal value of $\alpha$ for which the markov chain is able
	entirely explore the region and in this case out of four choices, it is $2$.
	\item On increasing the number of trajectories $L$ in Ensemble Methods, we can see from the plots \ref{fig:en_2_iat_20} and \ref{fig:en_2_iat} and from \ref{fig:en_1.5_iat_20} and \ref{fig:en_1.5_iat} that the IAT value decreases even for the same sample counts. So having more trajectories makes them iids quickly.
	\item Ensemble Scheme is not only better in terms of reuslts (as evident from attached images), it is simpler to implement as well. We don't need a lot of information regarding the
	differential equation governing the generation proces.
	\item Ensemble Schemes are much faster as well because of no need to compute hessian. Working with Hessian is a complex task, lot of thinks have to be taken care of like checkign for Symmetry, positive definiteness, invertibility, etc.
	Ensemble Scehems don't face such problems.
\end{enumerate}

Hence, we can easily conclude that Ensemble Method with $\alpha = 2$ is the best method for generating points from Rosenbrock Density.

\end{document}
